% Options for packages loaded elsewhere
\PassOptionsToPackage{unicode}{hyperref}
\PassOptionsToPackage{hyphens}{url}
%
\documentclass[
]{article}
\usepackage{amsmath,amssymb}
\usepackage{lmodern}
\usepackage{ifxetex,ifluatex}
\ifnum 0\ifxetex 1\fi\ifluatex 1\fi=0 % if pdftex
  \usepackage[T1]{fontenc}
  \usepackage[utf8]{inputenc}
  \usepackage{textcomp} % provide euro and other symbols
\else % if luatex or xetex
  \usepackage{unicode-math}
  \defaultfontfeatures{Scale=MatchLowercase}
  \defaultfontfeatures[\rmfamily]{Ligatures=TeX,Scale=1}
\fi
% Use upquote if available, for straight quotes in verbatim environments
\IfFileExists{upquote.sty}{\usepackage{upquote}}{}
\IfFileExists{microtype.sty}{% use microtype if available
  \usepackage[]{microtype}
  \UseMicrotypeSet[protrusion]{basicmath} % disable protrusion for tt fonts
}{}
\makeatletter
\@ifundefined{KOMAClassName}{% if non-KOMA class
  \IfFileExists{parskip.sty}{%
    \usepackage{parskip}
  }{% else
    \setlength{\parindent}{0pt}
    \setlength{\parskip}{6pt plus 2pt minus 1pt}}
}{% if KOMA class
  \KOMAoptions{parskip=half}}
\makeatother
\usepackage{xcolor}
\IfFileExists{xurl.sty}{\usepackage{xurl}}{} % add URL line breaks if available
\IfFileExists{bookmark.sty}{\usepackage{bookmark}}{\usepackage{hyperref}}
\hypersetup{
  pdftitle={Assignment\_1},
  hidelinks,
  pdfcreator={LaTeX via pandoc}}
\urlstyle{same} % disable monospaced font for URLs
\usepackage[margin=1in]{geometry}
\usepackage{color}
\usepackage{fancyvrb}
\newcommand{\VerbBar}{|}
\newcommand{\VERB}{\Verb[commandchars=\\\{\}]}
\DefineVerbatimEnvironment{Highlighting}{Verbatim}{commandchars=\\\{\}}
% Add ',fontsize=\small' for more characters per line
\usepackage{framed}
\definecolor{shadecolor}{RGB}{248,248,248}
\newenvironment{Shaded}{\begin{snugshade}}{\end{snugshade}}
\newcommand{\AlertTok}[1]{\textcolor[rgb]{0.94,0.16,0.16}{#1}}
\newcommand{\AnnotationTok}[1]{\textcolor[rgb]{0.56,0.35,0.01}{\textbf{\textit{#1}}}}
\newcommand{\AttributeTok}[1]{\textcolor[rgb]{0.77,0.63,0.00}{#1}}
\newcommand{\BaseNTok}[1]{\textcolor[rgb]{0.00,0.00,0.81}{#1}}
\newcommand{\BuiltInTok}[1]{#1}
\newcommand{\CharTok}[1]{\textcolor[rgb]{0.31,0.60,0.02}{#1}}
\newcommand{\CommentTok}[1]{\textcolor[rgb]{0.56,0.35,0.01}{\textit{#1}}}
\newcommand{\CommentVarTok}[1]{\textcolor[rgb]{0.56,0.35,0.01}{\textbf{\textit{#1}}}}
\newcommand{\ConstantTok}[1]{\textcolor[rgb]{0.00,0.00,0.00}{#1}}
\newcommand{\ControlFlowTok}[1]{\textcolor[rgb]{0.13,0.29,0.53}{\textbf{#1}}}
\newcommand{\DataTypeTok}[1]{\textcolor[rgb]{0.13,0.29,0.53}{#1}}
\newcommand{\DecValTok}[1]{\textcolor[rgb]{0.00,0.00,0.81}{#1}}
\newcommand{\DocumentationTok}[1]{\textcolor[rgb]{0.56,0.35,0.01}{\textbf{\textit{#1}}}}
\newcommand{\ErrorTok}[1]{\textcolor[rgb]{0.64,0.00,0.00}{\textbf{#1}}}
\newcommand{\ExtensionTok}[1]{#1}
\newcommand{\FloatTok}[1]{\textcolor[rgb]{0.00,0.00,0.81}{#1}}
\newcommand{\FunctionTok}[1]{\textcolor[rgb]{0.00,0.00,0.00}{#1}}
\newcommand{\ImportTok}[1]{#1}
\newcommand{\InformationTok}[1]{\textcolor[rgb]{0.56,0.35,0.01}{\textbf{\textit{#1}}}}
\newcommand{\KeywordTok}[1]{\textcolor[rgb]{0.13,0.29,0.53}{\textbf{#1}}}
\newcommand{\NormalTok}[1]{#1}
\newcommand{\OperatorTok}[1]{\textcolor[rgb]{0.81,0.36,0.00}{\textbf{#1}}}
\newcommand{\OtherTok}[1]{\textcolor[rgb]{0.56,0.35,0.01}{#1}}
\newcommand{\PreprocessorTok}[1]{\textcolor[rgb]{0.56,0.35,0.01}{\textit{#1}}}
\newcommand{\RegionMarkerTok}[1]{#1}
\newcommand{\SpecialCharTok}[1]{\textcolor[rgb]{0.00,0.00,0.00}{#1}}
\newcommand{\SpecialStringTok}[1]{\textcolor[rgb]{0.31,0.60,0.02}{#1}}
\newcommand{\StringTok}[1]{\textcolor[rgb]{0.31,0.60,0.02}{#1}}
\newcommand{\VariableTok}[1]{\textcolor[rgb]{0.00,0.00,0.00}{#1}}
\newcommand{\VerbatimStringTok}[1]{\textcolor[rgb]{0.31,0.60,0.02}{#1}}
\newcommand{\WarningTok}[1]{\textcolor[rgb]{0.56,0.35,0.01}{\textbf{\textit{#1}}}}
\usepackage{longtable,booktabs,array}
\usepackage{calc} % for calculating minipage widths
% Correct order of tables after \paragraph or \subparagraph
\usepackage{etoolbox}
\makeatletter
\patchcmd\longtable{\par}{\if@noskipsec\mbox{}\fi\par}{}{}
\makeatother
% Allow footnotes in longtable head/foot
\IfFileExists{footnotehyper.sty}{\usepackage{footnotehyper}}{\usepackage{footnote}}
\makesavenoteenv{longtable}
\usepackage{graphicx}
\makeatletter
\def\maxwidth{\ifdim\Gin@nat@width>\linewidth\linewidth\else\Gin@nat@width\fi}
\def\maxheight{\ifdim\Gin@nat@height>\textheight\textheight\else\Gin@nat@height\fi}
\makeatother
% Scale images if necessary, so that they will not overflow the page
% margins by default, and it is still possible to overwrite the defaults
% using explicit options in \includegraphics[width, height, ...]{}
\setkeys{Gin}{width=\maxwidth,height=\maxheight,keepaspectratio}
% Set default figure placement to htbp
\makeatletter
\def\fps@figure{htbp}
\makeatother
\setlength{\emergencystretch}{3em} % prevent overfull lines
\providecommand{\tightlist}{%
  \setlength{\itemsep}{0pt}\setlength{\parskip}{0pt}}
\setcounter{secnumdepth}{-\maxdimen} % remove section numbering
\ifluatex
  \usepackage{selnolig}  % disable illegal ligatures
\fi

\title{Assignment\_1}
\author{}
\date{\vspace{-2.5em}1/23/2022}

\begin{document}
\maketitle

First, load in the required libraries and the data.

\begin{Shaded}
\begin{Highlighting}[]
\FunctionTok{library}\NormalTok{(tidyverse)}
\NormalTok{bc\_data }\OtherTok{=} \FunctionTok{read.csv}\NormalTok{(}\StringTok{"bcdata.csv"}\NormalTok{)}
\end{Highlighting}
\end{Shaded}

\hypertarget{question-1}{%
\subsubsection{Question \#1}\label{question-1}}

Construct a table providing summaries of the quantitative features of
the dataset.Summaries should include the mean, median, minimum value,
and maximum value.

\begin{Shaded}
\begin{Highlighting}[]
\NormalTok{summary\_data }\OtherTok{=} \FunctionTok{summary}\NormalTok{(bc\_data)}
\FunctionTok{as.data.frame.matrix}\NormalTok{(summary\_data) }\SpecialCharTok{\%\textgreater{}\%} 
\NormalTok{  knitr}\SpecialCharTok{::}\FunctionTok{kable}\NormalTok{()}
\end{Highlighting}
\end{Shaded}

\begin{longtable}[]{@{}
  >{\raggedright\arraybackslash}p{(\columnwidth - 20\tabcolsep) * \real{0.03}}
  >{\raggedright\arraybackslash}p{(\columnwidth - 20\tabcolsep) * \real{0.08}}
  >{\raggedright\arraybackslash}p{(\columnwidth - 20\tabcolsep) * \real{0.09}}
  >{\raggedright\arraybackslash}p{(\columnwidth - 20\tabcolsep) * \real{0.10}}
  >{\raggedright\arraybackslash}p{(\columnwidth - 20\tabcolsep) * \real{0.10}}
  >{\raggedright\arraybackslash}p{(\columnwidth - 20\tabcolsep) * \real{0.10}}
  >{\raggedright\arraybackslash}p{(\columnwidth - 20\tabcolsep) * \real{0.10}}
  >{\raggedright\arraybackslash}p{(\columnwidth - 20\tabcolsep) * \real{0.10}}
  >{\raggedright\arraybackslash}p{(\columnwidth - 20\tabcolsep) * \real{0.10}}
  >{\raggedright\arraybackslash}p{(\columnwidth - 20\tabcolsep) * \real{0.10}}
  >{\raggedright\arraybackslash}p{(\columnwidth - 20\tabcolsep) * \real{0.10}}@{}}
\toprule
& Age & BMI & Glucose & Insulin & HOMA & Leptin & Adiponectin & Resistin
& MCP.1 & Classification \\
\midrule
\endhead
X & Min. :24.0 & Min. :18.37 & Min. : 60.00 & Min. : 2.432 & Min. :
0.4674 & Min. : 4.311 & Min. : 1.656 & Min. : 3.210 & Min. : 45.84 &
Min. :1.000 \\
X.1 & 1st Qu.:45.0 & 1st Qu.:22.97 & 1st Qu.: 85.75 & 1st Qu.: 4.359 &
1st Qu.: 0.9180 & 1st Qu.:12.314 & 1st Qu.: 5.474 & 1st Qu.: 6.882 & 1st
Qu.: 269.98 & 1st Qu.:1.000 \\
X.2 & Median :56.0 & Median :27.66 & Median : 92.00 & Median : 5.925 &
Median : 1.3809 & Median :20.271 & Median : 8.353 & Median :10.828 &
Median : 471.32 & Median :2.000 \\
X.3 & Mean :57.3 & Mean :27.58 & Mean : 97.79 & Mean :10.012 & Mean :
2.6950 & Mean :26.615 & Mean :10.181 & Mean :14.726 & Mean : 534.65 &
Mean :1.552 \\
X.4 & 3rd Qu.:71.0 & 3rd Qu.:31.24 & 3rd Qu.:102.00 & 3rd Qu.:11.189 &
3rd Qu.: 2.8578 & 3rd Qu.:37.378 & 3rd Qu.:11.816 & 3rd Qu.:17.755 & 3rd
Qu.: 700.09 & 3rd Qu.:2.000 \\
X.5 & Max. :89.0 & Max. :38.58 & Max. :201.00 & Max. :58.460 & Max.
:25.0503 & Max. :90.280 & Max. :38.040 & Max. :82.100 & Max. :1698.44 &
Max. :2.000 \\
\bottomrule
\end{longtable}

\hypertarget{question-2}{%
\subsubsection{Question 2}\label{question-2}}

Recode BMI into the WHO-defined categories

\begin{Shaded}
\begin{Highlighting}[]
\NormalTok{new\_data }\OtherTok{=}\NormalTok{ bc\_data }\SpecialCharTok{\%\textgreater{}\%}
  \FunctionTok{mutate}\NormalTok{(}
    \AttributeTok{BMI =} \FunctionTok{case\_when}\NormalTok{(}
\NormalTok{    BMI }\SpecialCharTok{\textless{}} \FloatTok{16.5} \SpecialCharTok{\textasciitilde{}} \StringTok{"Severely underweight"}\NormalTok{,}
\NormalTok{    BMI }\SpecialCharTok{\textgreater{}} \FloatTok{16.5} \SpecialCharTok{\&}\NormalTok{ BMI }\SpecialCharTok{\textless{}} \FloatTok{18.5} \SpecialCharTok{\textasciitilde{}} \StringTok{"Underweight"}\NormalTok{,}
\NormalTok{    BMI }\SpecialCharTok{\textgreater{}=} \FloatTok{18.5} \SpecialCharTok{\&}\NormalTok{ BMI }\SpecialCharTok{\textless{}=} \FloatTok{24.9} \SpecialCharTok{\textasciitilde{}} \StringTok{"Normal weight"}\NormalTok{,      }
\NormalTok{    BMI }\SpecialCharTok{\textgreater{}=} \DecValTok{25} \SpecialCharTok{\&}\NormalTok{ BMI }\SpecialCharTok{\textless{}=} \FloatTok{29.9} \SpecialCharTok{\textasciitilde{}} \StringTok{"Overweight"}\NormalTok{,}
\NormalTok{    BMI }\SpecialCharTok{\textgreater{}=} \DecValTok{30} \SpecialCharTok{\&}\NormalTok{ BMI }\SpecialCharTok{\textless{}=} \FloatTok{34.9} \SpecialCharTok{\textasciitilde{}} \StringTok{"Obsesity class I"}\NormalTok{,}
\NormalTok{    BMI }\SpecialCharTok{\textgreater{}=} \DecValTok{35} \SpecialCharTok{\&}\NormalTok{ BMI }\SpecialCharTok{\textless{}=}\FloatTok{39.9} \SpecialCharTok{\textasciitilde{}} \StringTok{"Obesity class II"}\NormalTok{,}
    \ConstantTok{TRUE} \SpecialCharTok{\textasciitilde{}} \StringTok{"Obesity class III"}\NormalTok{)}
\NormalTok{    ) }\SpecialCharTok{\%\textgreater{}\%} 
  \FunctionTok{mutate}\NormalTok{(}
    \AttributeTok{BMI =} \FunctionTok{factor}\NormalTok{(BMI, }\AttributeTok{levels =} \FunctionTok{c}\NormalTok{(}\StringTok{"Severely underweight"}\NormalTok{, }\StringTok{"Underweight"}\NormalTok{, }\StringTok{"Normal weight"}\NormalTok{, }\StringTok{"Overweight"}\NormalTok{, }\StringTok{"Obsesity class I"}\NormalTok{, }\StringTok{"Obesity class II"}\NormalTok{, }\StringTok{"Obesity class III"}\NormalTok{))}
\NormalTok{  )}

\FunctionTok{str}\NormalTok{(new\_data}\SpecialCharTok{$}\NormalTok{BMI)}
\end{Highlighting}
\end{Shaded}

\begin{verbatim}
##  Factor w/ 7 levels "Severely underweight",..: 3 3 3 3 3 3 3 3 3 3 ...
\end{verbatim}

\hypertarget{question-3}{%
\subsubsection{Question 3}\label{question-3}}

Create a bar chart showing the proportion of breast cancer cases and
controls within each BMI category.

\hypertarget{proportion-of-breast-cancer}{%
\paragraph{Proportion of Breast
Cancer}\label{proportion-of-breast-cancer}}

\begin{Shaded}
\begin{Highlighting}[]
\NormalTok{new\_data2 }\OtherTok{=}
\NormalTok{new\_data }\SpecialCharTok{\%\textgreater{}\%} 
  \FunctionTok{group\_by}\NormalTok{(BMI) }\SpecialCharTok{\%\textgreater{}\%} 
  \FunctionTok{summarize}\NormalTok{(}
    \AttributeTok{control =} \FunctionTok{sum}\NormalTok{(Classification }\SpecialCharTok{==} \DecValTok{1}\NormalTok{) }\SpecialCharTok{/} \FunctionTok{n}\NormalTok{(),}
    \AttributeTok{bc =} \FunctionTok{sum}\NormalTok{(Classification }\SpecialCharTok{==} \DecValTok{2}\NormalTok{)}\SpecialCharTok{/} \FunctionTok{n}\NormalTok{()}
\NormalTok{  ) }\SpecialCharTok{\%\textgreater{}\%} 
  \FunctionTok{ggplot}\NormalTok{(}\FunctionTok{aes}\NormalTok{(}\AttributeTok{x =}\NormalTok{ BMI, }\AttributeTok{y =}\NormalTok{ bc, }\AttributeTok{fill =}\NormalTok{ BMI)) }\SpecialCharTok{+}
  \FunctionTok{geom\_bar}\NormalTok{(}\AttributeTok{stat =} \StringTok{"identity"}\NormalTok{) }\SpecialCharTok{+}
  \FunctionTok{ggtitle}\NormalTok{(}\StringTok{"Proportion of breast cancer cases in each BMI category"}\NormalTok{) }\SpecialCharTok{+}
  \FunctionTok{xlab}\NormalTok{(}\StringTok{"BMI category"}\NormalTok{) }\SpecialCharTok{+} \FunctionTok{ylab}\NormalTok{(}\StringTok{"Proportion of cases"}\NormalTok{)}

\NormalTok{new\_data2}
\end{Highlighting}
\end{Shaded}

\includegraphics{assignment_01_files/figure-latex/unnamed-chunk-4-1.pdf}

\hypertarget{proportion-of-control}{%
\paragraph{Proportion of Control}\label{proportion-of-control}}

\begin{Shaded}
\begin{Highlighting}[]
\NormalTok{new\_data3 }\OtherTok{=}
\NormalTok{new\_data }\SpecialCharTok{\%\textgreater{}\%} 
  \FunctionTok{group\_by}\NormalTok{(BMI) }\SpecialCharTok{\%\textgreater{}\%} 
  \FunctionTok{summarize}\NormalTok{(}
    \AttributeTok{control =} \FunctionTok{sum}\NormalTok{(Classification }\SpecialCharTok{==} \DecValTok{1}\NormalTok{) }\SpecialCharTok{/} \FunctionTok{n}\NormalTok{(),}
    \AttributeTok{bc =} \FunctionTok{sum}\NormalTok{(Classification }\SpecialCharTok{==} \DecValTok{2}\NormalTok{)}\SpecialCharTok{/} \FunctionTok{n}\NormalTok{()}
\NormalTok{  ) }\SpecialCharTok{\%\textgreater{}\%} 
  \FunctionTok{ggplot}\NormalTok{(}\FunctionTok{aes}\NormalTok{(}\AttributeTok{x =}\NormalTok{ BMI, }\AttributeTok{y =}\NormalTok{ control, }\AttributeTok{fill =}\NormalTok{ BMI)) }\SpecialCharTok{+}
  \FunctionTok{geom\_bar}\NormalTok{(}\AttributeTok{stat =} \StringTok{"identity"}\NormalTok{) }\SpecialCharTok{+} 
  \FunctionTok{ggtitle}\NormalTok{(}\StringTok{"proportion of controls in each BMI category"}\NormalTok{) }\SpecialCharTok{+}
  \FunctionTok{xlab}\NormalTok{(}\StringTok{"BMI category"}\NormalTok{) }\SpecialCharTok{+} \FunctionTok{ylab}\NormalTok{(}\StringTok{"proportion of controls"}\NormalTok{)}

\NormalTok{new\_data3}
\end{Highlighting}
\end{Shaded}

\includegraphics{assignment_01_files/figure-latex/unnamed-chunk-5-1.pdf}

\hypertarget{question-4}{%
\subsubsection{Question 4}\label{question-4}}

Construct a \textbf{logistic} regression model:

\begin{itemize}
\tightlist
\item
  outcome: breast cancer classification\\
\item
  independent variables: glucose, HOMA-IR, leptin, BMI (continuous), age
\end{itemize}

Fill in the beta estimate and 95\% confidence interval associated with a
1-unit change in HOMA-IR.

\begin{Shaded}
\begin{Highlighting}[]
\NormalTok{model\_data }\OtherTok{=}\NormalTok{ bc\_data }\SpecialCharTok{\%\textgreater{}\%} 
\NormalTok{  janitor}\SpecialCharTok{::}\FunctionTok{clean\_names}\NormalTok{() }\SpecialCharTok{\%\textgreater{}\%} 
  \FunctionTok{mutate}\NormalTok{(}
    \AttributeTok{classification =} \FunctionTok{ifelse}\NormalTok{(classification }\SpecialCharTok{==} \DecValTok{1}\NormalTok{, }\DecValTok{0}\NormalTok{, }\DecValTok{1}\NormalTok{)}
\NormalTok{  )}

\NormalTok{logistic\_model }\OtherTok{=} \FunctionTok{glm}\NormalTok{(classification }\SpecialCharTok{\textasciitilde{}}\NormalTok{ glucose }\SpecialCharTok{+}\NormalTok{ homa }\SpecialCharTok{+}\NormalTok{ leptin }\SpecialCharTok{+}\NormalTok{ bmi }\SpecialCharTok{+}\NormalTok{ age, }\AttributeTok{data =}\NormalTok{ model\_data, }\AttributeTok{family =} \StringTok{"binomial"}\NormalTok{)}

\FunctionTok{summary}\NormalTok{(logistic\_model)}
\DocumentationTok{\#\# }
\DocumentationTok{\#\# Call:}
\DocumentationTok{\#\# glm(formula = classification \textasciitilde{} glucose + homa + leptin + bmi + }
\DocumentationTok{\#\#     age, family = "binomial", data = model\_data)}
\DocumentationTok{\#\# }
\DocumentationTok{\#\# Deviance Residuals: }
\DocumentationTok{\#\#     Min       1Q   Median       3Q      Max  }
\DocumentationTok{\#\# {-}2.2944  {-}0.8901   0.1308   0.8084   2.1371  }
\DocumentationTok{\#\# }
\DocumentationTok{\#\# Coefficients:}
\DocumentationTok{\#\#              Estimate Std. Error z value Pr(\textgreater{}|z|)    }
\DocumentationTok{\#\# (Intercept) {-}3.626065   2.355177  {-}1.540 0.123654    }
\DocumentationTok{\#\# glucose      0.081699   0.023526   3.473 0.000515 ***}
\DocumentationTok{\#\# homa         0.273882   0.171976   1.593 0.111259    }
\DocumentationTok{\#\# leptin      {-}0.008574   0.015783  {-}0.543 0.586979    }
\DocumentationTok{\#\# bmi         {-}0.104261   0.056642  {-}1.841 0.065668 .  }
\DocumentationTok{\#\# age         {-}0.022881   0.014377  {-}1.592 0.111496    }
\DocumentationTok{\#\# {-}{-}{-}}
\DocumentationTok{\#\# Signif. codes:  0 \textquotesingle{}***\textquotesingle{} 0.001 \textquotesingle{}**\textquotesingle{} 0.01 \textquotesingle{}*\textquotesingle{} 0.05 \textquotesingle{}.\textquotesingle{} 0.1 \textquotesingle{} \textquotesingle{} 1}
\DocumentationTok{\#\# }
\DocumentationTok{\#\# (Dispersion parameter for binomial family taken to be 1)}
\DocumentationTok{\#\# }
\DocumentationTok{\#\#     Null deviance: 159.57  on 115  degrees of freedom}
\DocumentationTok{\#\# Residual deviance: 120.81  on 110  degrees of freedom}
\DocumentationTok{\#\# AIC: 132.81}
\DocumentationTok{\#\# }
\DocumentationTok{\#\# Number of Fisher Scoring iterations: 6}
\FunctionTok{confint}\NormalTok{(logistic\_model) }
\DocumentationTok{\#\#                   2.5 \%      97.5 \%}
\DocumentationTok{\#\# (Intercept) {-}8.54138756 0.754487774}
\DocumentationTok{\#\# glucose      0.03956613 0.132397841}
\DocumentationTok{\#\# homa        {-}0.02555240 0.653222623}
\DocumentationTok{\#\# leptin      {-}0.04019445 0.022416142}
\DocumentationTok{\#\# bmi         {-}0.21944692 0.004398024}
\DocumentationTok{\#\# age         {-}0.05192184 0.004856327}
\end{Highlighting}
\end{Shaded}

For every one unit change in HOMA-IR, the log odds of having breast
cancer increases by \textbf{0.2738822} (95\% CI: -0.025, 0.653).

\url{https://stats.oarc.ucla.edu/r/dae/logit-regression/}

\hypertarget{question-5}{%
\subsubsection{Question 5}\label{question-5}}

Construct a \textbf{linear} regression model:

\begin{itemize}
\tightlist
\item
  outcome: insulin\\
\item
  independent variables: BMI (continuous), age, glucose
\end{itemize}

Fill in the beta estimate and 95\% confidence interval associated with a
1-unit change in age.

\begin{Shaded}
\begin{Highlighting}[]
\NormalTok{linear\_model }\OtherTok{=} \FunctionTok{lm}\NormalTok{(insulin }\SpecialCharTok{\textasciitilde{}}\NormalTok{ bmi }\SpecialCharTok{+}\NormalTok{ age }\SpecialCharTok{+}\NormalTok{ glucose, }\AttributeTok{data =}\NormalTok{ model\_data)}
\FunctionTok{summary}\NormalTok{(linear\_model)}
\DocumentationTok{\#\# }
\DocumentationTok{\#\# Call:}
\DocumentationTok{\#\# lm(formula = insulin \textasciitilde{} bmi + age + glucose, data = model\_data)}
\DocumentationTok{\#\# }
\DocumentationTok{\#\# Residuals:}
\DocumentationTok{\#\#     Min      1Q  Median      3Q     Max }
\DocumentationTok{\#\# {-}22.161  {-}4.359  {-}2.118   2.124  46.269 }
\DocumentationTok{\#\# }
\DocumentationTok{\#\# Coefficients:}
\DocumentationTok{\#\#              Estimate Std. Error t value Pr(\textgreater{}|t|)    }
\DocumentationTok{\#\# (Intercept) {-}13.49576    5.85941  {-}2.303   0.0231 *  }
\DocumentationTok{\#\# bmi           0.14969    0.16382   0.914   0.3628    }
\DocumentationTok{\#\# age          {-}0.05402    0.05194  {-}1.040   0.3005    }
\DocumentationTok{\#\# glucose       0.22982    0.03752   6.126 1.37e{-}08 ***}
\DocumentationTok{\#\# {-}{-}{-}}
\DocumentationTok{\#\# Signif. codes:  0 \textquotesingle{}***\textquotesingle{} 0.001 \textquotesingle{}**\textquotesingle{} 0.01 \textquotesingle{}*\textquotesingle{} 0.05 \textquotesingle{}.\textquotesingle{} 0.1 \textquotesingle{} \textquotesingle{} 1}
\DocumentationTok{\#\# }
\DocumentationTok{\#\# Residual standard error: 8.731 on 112 degrees of freedom}
\DocumentationTok{\#\# Multiple R{-}squared:  0.2675, Adjusted R{-}squared:  0.2479 }
\DocumentationTok{\#\# F{-}statistic: 13.64 on 3 and 112 DF,  p{-}value: 1.207e{-}07}
\FunctionTok{confint}\NormalTok{(linear\_model)}
\DocumentationTok{\#\#                   2.5 \%      97.5 \%}
\DocumentationTok{\#\# (Intercept) {-}25.1054353 {-}1.88608318}
\DocumentationTok{\#\# bmi          {-}0.1748942  0.47427491}
\DocumentationTok{\#\# age          {-}0.1569321  0.04888876}
\DocumentationTok{\#\# glucose       0.1554864  0.30414939}
\end{Highlighting}
\end{Shaded}

For every one year increase in age, the insulin decreases by
\textbf{-0.0540217} microU/mL (95\% CI: -0.157, 0.048).

\end{document}
